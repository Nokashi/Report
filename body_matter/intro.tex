\chapter{Introduction} 


\InitialCharacter{T} he advent of IoT devices, such as sensors, smartwatches, smartphones, and GPS trackers, has led to the production of a huge volume of data, including mobility data, on a daily basis, a quantity that will continue to grow over time. These data sets, after appropriate processing, are critical for understanding mobility patterns \cite{DBLP:conf/edbt/TritsarolisCTP21, DBLP:journals/geoinformatica/TritsarolisCTPT24}, which allow conclusions to be drawn, but also predictions of significant importance for multiple sectors \cite{VourosBigDataAnalytics}, such as ship/fleet tracking \cite{DBLP:conf/mdm/TritsarolisPBZT24}, route risk assessment \cite{DBLP:conf/gis/MichalskiPRTTP24}, traffic flow prediction \cite{DBLP:conf/edbt/MandalisCKPT23}, ride-sharing applications, etc.

This volume of information makes it necessary to create a workflow that will be able to bridge the gap between "pure" geographic information and human understanding of the results that an analysis of this information can offer. This need is addressed by the research field of Visual Analytics (VA), which creates a framework in which methods and software exploit human "vision" and decision-making to extract knowledge from spatiotemporal data \cite{vision2020}, both of movement (Mobility Data) \cite{MaSEC,DBLP:conf/gis/TritsarolisPT23}, as well as other types (Social Networks, Economy, Climate Science) \cite{SensePlace3,EconomicVA,nayeem2024}.

Therefore, there is a need for easy support for Visual Analytics applications on data, so that the respective domain experts can have an overview of the data they are working on. For these purposes, the Data Science Laboratory of the University of Piraeus \footnote{\url{https://www.datastories.org}} created the ST-Visions \cite{st_visions} library for visualizing spatiotemporal data. The purpose of this bachelor's thesis is to enhance the capabilities of the library by adding live data streaming visualization support. This has the aim of enabling the visualization and analysis of spatiotemporal data in real time (Real-Time Domain Visual Analytics). 

\section{Problem Statement}

Along with the sheer quantity of available static mobility data, technological advancements in navigation satellite systems has brought about an exponential increase in mobility data streams that are produced continuously in real time\cite{galic2017bigdata}





\section{Structure of the 2nd Progress Report}

